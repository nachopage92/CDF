\documentclass[letterpaper]{article}
\usepackage{amsmath}
\usepackage{amsfonts}
\usepackage{fontspec}
\usepackage{graphicx}
\usepackage{float}		%\begin{figure}[H] <- agrega el H
\usepackage{multirow}
\usepackage{multicol}
\usepackage{indentfirst}
\usepackage{caption} %comando \ContinuedFloat
\usepackage{array} %paquete para tabular
\usepackage{subcaption} %subfiguras y continuación de figura
\usepackage{pstricks-add}
\usepackage{bm}
\usepackage{enumitem} %cara utilizar distintos enumerate item


\newcolumntype{P}[1]{>{\centering\arraybackslash}p{#1}} % P{} (mayúscula) en vez de p{} (minúscula)
\newcommand{\vect}[1]{\boldsymbol{#1}} %notación vector \vect{•}
\renewcommand{\contentsname}{Índice}
\renewcommand{\figurename}{Figura}
\renewcommand{\tablename}{Tabla}
\renewcommand\refname{Referencias}

%%% quita los ":" del \caption
\makeatletter
\long\def\@makecaption#1#2{%
\vskip\abovecaptionskip
\sbox\@tempboxa{#1. #2}%
\ifdim \wd\@tempboxa >\hsize
#1. #2\par
\else
\global \@minipagefalse
\hb@xt@\hsize{\hfil\box\@tempboxa\hfil}%
\fi
\vskip\belowcaptionskip}
\makeatother
%%%%%%%%%%%%%%%%%%%%%%%%%


\pagestyle{plain}


%FORMATO PÁGINA
%\hoffset
\voffset=-2cm
\oddsidemargin=0.8cm
%\evensidemargin
%\topmargin 		%entre sup y encab
%\headheight		%tamaño encabezado
%\headsep			%sep encab y text
\textheight=23cm		%altura texto
\textwidth=15cm		%ancho texto
%\marginparsep	%sep notmargen y text
%\marginparwidth	%ancho nota al marg
\footskip=1cm		%pie de pag

\begin{document}

\thispagestyle{empty}

\hspace{-5mm}
\begin{minipage}[c]{7cm}
\centering
\includegraphics[width=4cm]{logoutfsm.jpg} \\
Universidad Técnica Federico Santa María
\end{minipage}
\hfill
\hspace{20mm}
\begin{minipage}[c]{7cm}
\centering
\includegraphics[width=4cm]{logomec1.jpg} \\
Departamento de Ingenieria Mecánica
\end{minipage}

\begin{center}
\vfill
 \Huge{{\bf Proyecto 1 }} \\ \vspace{1cm} 
 \Huge{Dinámica de fluidos computacional}
\vfill
\end{center}

\vfill \hfill
\begin{tabular}{l @{ : } l}
Nombre &   Ignacio Apablaza \\
Rol & 201141007-6  \\
Profesores & Romain Gers \\
			& Olivier Skurtys \\
Asignatura & IPM468 \\
\end{tabular}

\newpage
%---------------------------------------------


\tableofcontents


\newpage
%---------------------------------------------

\section{Resumen}






\newpage
%---------------------------------------------

\section{Introducción}







\newpage
%---------------------------------------------

\section{Metodología}
%METODOLOGÍA

%-------------------------------------------------------------- 

\subsection{Precisión numérica en Fortran}
Fortran (\textit{Formula Translator} o \textit{Traductor de Fórmulas}) es un lenguaje de programación orientado a objetos y de alto nivel utilizado para la computación científica en distintas disciplinas del área de las ciencias. Fortran posee distintos tipos de objetos:

\begin{description}
\item [character] cadena de uno o varios caracteres \vspace{-0,2cm}
\item [integer] números enteros positivos y negativos \vspace{-0,2cm}
\item [logical] valores lógicos o booleanos (\texttt{.true.} o \texttt{.false.})\vspace{-0,2cm}
\item [real] números reales positivos y negativos  \vspace{-0,2cm}
\item [complex] números complejos compuestos de una parte real y una imaginaria \vspace{-0,2cm}
\item [tipos derivados] tipos especificados por usuario
\end{description}

Los objetos de clase \texttt{real} poseen ciertos parámetros que describen sus características. Un paramétro relevante a estudiar es la precisión que describe a un objeto declarado como \texttt{real}

\begin{center}
\begin{table} [H]
\begin{tabular}{lll}
Entero & $−2.147.483.648 \leq i \leq 2.147.483.647$ & $-$ \\
Real Simple Precisión & $1.2 \times 10^{−38} \leq |x| \leq 3.4 \times 10^{38}$  & 7 cifras significativas\\
Real Doble Precision & $2.2 \times 10^{−308} \leq |x| \leq 1.8 \times 10^{308}$ & 16 cifras significativas
\end{tabular}
\caption{Características de precisión de reales en Fortran} \label{TABLA_FORTRAN}
\end{table}
\end{center}

La especificación del parámetro precisión especificará el tamaño de memoria asignada al objeto. Dependiendo de la naturaleza del cálculo empleado será más conveniente utilizar una u otra precisión.

%--------------------------------------------------------------
%--------------------------------------------------------------
%--------------------------------------------------------------
%-------------------------------------------------------------- 

\subsection{Método de Diferencias Finitas}

Una manera de aproximar numéricamente derivadas presentes en una ecuación diferencial ordinaria o parcial es mediante el método de Diferencias Finitas, que consiste representar las razones de cambio como una diferencia de valores nodales discretos. Se desprende del desarrollo de series de Taylor de la función incógnita: Sea $\phi(x)$ una función diferenciales en una dimensión en un dominio de interés, entonces el valor de $\phi(x + \Delta x)$ se puede expresar mediante el desarrollo en series de Taylor:

\begin{equation} \label{serie_taylor}
\phi( x + \Delta x ) = \sum_{n=0}^{\infty} \dfrac{\Delta x^n}{n!} \dfrac{ \partial \phi^{(n)} } { \partial ^n} \Big|_x
\end{equation}
Para una diferencia $+\Delta x$ se tiene,
\begin{equation} \label{delta+}
\phi( x + \Delta x ) = \phi(x) + \Delta x \dfrac{\partial \phi }{\partial x} \Big|_x + \dfrac{  (\Delta x)^2 }{2} \dfrac{\partial^2 \phi }{\partial x^2} \Big|_x + \dfrac{ (\Delta x)^3 }{6} \dfrac{\partial^3 \phi }{\partial x^3} \Big|_x + \cdots
\end{equation}
Para una diferencia $-\Delta x$ se tiene
\begin{equation} \label{delta-}
\phi( x - \Delta x ) = \phi(x) - \Delta x \dfrac{\partial \phi }{\partial x} \Big|_x + \dfrac{  (\Delta x)^2 }{2} \dfrac{\partial^2 \phi }{\partial x^2} \Big|_x - \dfrac{ (\Delta x)^3 }{6} \dfrac{\partial^3 \phi }{\partial x^3} \Big|_x + \cdots
\end{equation}
Distintas combinaciones de las ecuaciones (\ref{delta+}) y (\ref{delta-}) permiten obtener las aproximaciones de distintos ordendes de derivada.\\

Truncando el desarrollo de la serie se obtiene la aproximación. Al conocer la expresión analitica en (\ref{serie_taylor}) se puede determinar el orden el error obtenido. El desarrollo en una dimensión se extiende a dimensiones superiores. A continuación se exponen los tipos de aproximaciones utilizados en este trabajo:

\paragraph{ Diferencia hacia atras (Backward) } Aproximando la primera derivada de $\phi(x)$ mediante diferencias hacia atras resulta: 
\begin{equation}
\dfrac{ \partial \phi }{ \partial x } \Big|_{x=x_i} = \dfrac{ \phi(x_i) - \phi(x_{i-1}) }{\Delta x} + o(\Delta x ^ 1)
\end{equation} 
$o(\Delta x)$ agrupa el términos truncados de la serie y representa el error de la aproximación, de tal manera que,
\begin{equation}
\dfrac{ \partial \phi _i }{ \partial x } \approx \dfrac{ \phi_i - \phi_{i-1} }{\Delta x}
\end{equation} 
Donde $\phi_i$ denota el valor nodal que discretiza a la función en el dominio ($\phi(x_i) = \phi_i$) , En este caso se tiene un error de orden 1

\paragraph{Diferencias centradas } Aproximando la segunda derivada de $\phi(x)$ utilizando diferencias finitas centradas resulta:
\begin{equation}
\dfrac{\partial^2 \phi}{\partial x^2} \Big| _{x_i} = \dfrac{ \phi(x_{i+1}) - 2 \phi (x_i) +\phi(x_{i-1}) }{ \Delta x^2 } + o(\Delta x^2 )
\end{equation}
Este esquema de aproximación posee un error de orden 2. Analogo al caso anterior la aproximación se plantea como,
\begin{equation}
\dfrac{\partial^2 \phi}{\partial x^2} \Big| _{x_i} \approx \dfrac{ \phi_{i+1} - 2 \phi_i +\phi_{i-1} }{ \Delta x^2 }
\end{equation}

El método de diferencias finitas se aplica para discretizar derivadas espaciales y temporales. Estas últimas determinan los esquemas de integración temporales.

\subsection{Esquema de integración temporal}

Sea una ecuación diferencial de $\phi(t)$ tal que,
\begin{equation}
\left\{
\begin{matrix}
\dfrac{d \phi(t)}{dt} = f(t,\phi(t)) \\
\phi(t=0) = \phi_0
\end{matrix}
\right.
\end{equation}
Se tiene un problema de Cauchy o de valor inicial. La solución de $\phi$ está dada por,
\begin{equation}
\int_{t_n}^{t_{n+1}} \dfrac{d \phi(t)}{dt} = \int_{t_n}^{t_{n+1}} f(t,\phi(t)) dt
\end{equation}
Del teorema fundamental del cálculo se tiene,
\begin{equation} \label{integracion_temporal}
\phi(t_{n+1}) - \phi({t_n})  = \int_{t_n}^{t_{n+1}}  f(t,\phi(t)) dt
\end{equation}

Según como se integre el término de la derecha de la ecuación en (\ref{integracion_temporal}) se obtienen los distintos esquemas de integración. 

\subsubsection{Esquema Euler Implicito}
El esquema de Euler Implicito (Backward) se define como,
\begin{equation}
\phi^{n+1} = \phi^n + \Delta t f(t_{n+1},\phi(t_{n+1}))
\end{equation}
Este esquema requiere conocer el valor del pasos $t_{n}$ y $t_{n+1}$. Este tipo de esquemas son conocidos como esquemas de dos pasos (\textit{two level scheme})

\subsubsection{Esquema integración de $\Theta$}
La familia de esquemas $\Theta$ se describen como,
\begin{equation}
\phi^{n+1} = \phi^n + \Theta f(t_{n},\phi(t_{n}) + (1-\Theta) f(t_{n+1},\phi(t_{n+1})
\end{equation}
Para el valor de $\Theta = \frac{1}{2}$ se tiene el esquema de Crank Nicolson:
\begin{equation}
\phi^{n+1} = \phi^n + \dfrac{\Delta t}{2} \left( f(t_{n},\phi(t_{n}) + f(t_{n+1},\phi(t_{n+1}) \right)
\end{equation}

\subsubsection{Esquema Leap-Frog}
Los esquemas Leap-Frog corresponden a una discretizacion central de la derivada temporal.
\begin{equation}
\phi^{n+1} = \phi^{n-1} + 2 \Delta t f(t_n,\phi(t_n))
\end{equation}
Se calcula el valor de $\phi$ en el tiempo $t_{n+1}$ a partir de dos valores anteriores $t_n$ y $t_{n-1}$. Estos tipos de esquema son llamadas de tres pasos (\textit{three level scheme})

\subsubsection{Esquema Newmark}
Consiste en un esquema de dos pasos para el cálculo de $\phi$ y su derivada $\partial \phi (t) / \partial t$. Sea $\psi(t) = \partial \phi(t) / \partial t$, se integra $\psi$ utilizando un esquema $\Theta$,
\begin{equation}
\psi_{n+1} = \psi_n + \Delta t \left[ \Theta \dfrac{\partial \psi}{\partial t}^{n} \Big|_t + (1-\Theta) \dfrac{\partial \psi}{\partial t}^{n+1} \Big|_t \right]
\end{equation}
Para integrar $\phi$ se utiliza un esquema explícito donde el último término utiliza un esquema $\Theta$
\begin{equation}
\phi_{n+1} = \phi_n + \Delta t \psi + (\Delta t)^2 \left[ \beta \dfrac{\partial \psi}{\partial t}^n \Big|_t + (1-\beta) \dfrac{\partial \psi}{\partial t}^{n+1} \Big|_t \right]
\end{equation}
donde $\beta$ es un parámetro que reemplaza a $\Theta$ e integra al factor $2$ del desarrollo de la serie de Taylor.

\subsubsection{Método Runge Kutta de orden 4}
Los métodos de Runge Kutta conocidos como métodos \textit{Predictor-Corrector}: Se calcula uno o varios valores intermedios de la función incógnita $\phi^*$, llamados predictores, para finalmente calcular el resultado final $\phi(t+\Delta t)$ (corrector). \\

El método Runge Kutta de orden 4 consiste en calcular tres pasos de predicción y el último paso de corrección:

\begin{itemize}
\item $1^{\mbox{ra}}$ predicción: Euler Explícito \vspace{-0,2cm}
\item $2^{\mbox{da}}$ predicción: Euler Implícito \vspace{-0,2cm}
\item $3^{\mbox{ra}}$ predicción: Leap-Frog \vspace{-0,2cm}
\item Corrección: Método de integración de Simpson 
\end{itemize}

Luego, se puede expresar como:

\begin{align}
\phi^*_{n+1/2} &= \phi_n + \dfrac{\Delta t}{2} f(t_n,\phi_n)\\
\phi^{**}_{n+1/2} &= \phi_n + \dfrac{\Delta t}{2} f(t_{n+1/2},\phi^*_{n+1/2})\\
\phi^*_{n+1} &= \phi_n + \Delta t f(t_{n+1/2},\phi^{**}_{n+1/2})\\
\phi_{n+1} &= \phi_n + \dfrac{\Delta t}{6} \left[ f(t_n,\phi_n) + 2 f(t_{n+1/2},\phi^*_{n+1/2}) + 2 f(t_{n+1/2},\phi^{**}_{n+1/2}) + f(t_{n+1},\phi^*_{n+1}) \right] 
\end{align}

\subsection{Analisis Espectral}
La discretización de una ecuación diferencial de $u=u(\vec{x},t)$ se puede expresar en una notación matricial: Sea $\vec{U}$ el vector que contiene los valores $u_i$ $ (i=1, \ldots ,n)$.
\begin{equation} \label{analisis_espectral_general}
\dfrac{d \vec{U}}{d t} = \vect{S} \vec{U} + \vec{Q}
\end{equation}
Donde $\vect{S}$ es la matriz asociada a la discretización espacial y $\vec{Q}$ es el vector que contiene los componentes del término fuente. Esta ecuación se descompone a partir de sus valores propios (Descomposición modal), en ella se desacopla la incognita en espacio y tiempo. Para cada componente del vector $\vec{U}$ se tiene,
\begin{equation} \label{analisis_espectral_descompuesto}
\dfrac{d \overline{U}_j}{d t} = \Omega_j \overline{U}_j + Q_j
\end{equation}
donde,
\begin{equation}
\vec{\overline{U}}(\vec{x},t) = \sum_{j=1}^N \vec{\overline{U}}_j(t) V^{(j)}(\vec{x})
\hspace{0,8cm} \mbox{y} \hspace{0,8cm}
Q = \sum_{j=1}^N Q_j V^{(j)}
\end{equation}
$\Omega_j$ son los autovalores asociados a la dirección del vector propio $V^{(j)}$ de la matriz $\vect{S}$; $N$ es el número de dimensiones de la ecuación (\ref{analisis_espectral_general}). Luego, su solución analítica en función de sus valores propios viene dado por,
\begin{equation}
\overline{U}_j(t) = \left( U^0 + \dfrac{Q_j}{\Omega_j} \right) e^{\Omega_j t} - \dfrac{Q_j}{\Omega_j}
\end{equation}
Donde $U^0$ es la condición inicial del problema de Cauchy. 

\subsubsection{Factor de Amplificación}

Es de interés conocer el comportamiento de la solución transiente de $\vec{U}$. Para ello se supone $Q=0$ (solución homogenea) y se denota como $U^T$ la solución transiente,
\begin{equation} \label{solución transiente}
U_j^T(t) = U^0 e^{\Omega_jt}
\end{equation}
Se define el factor de amplificación $G(\Omega_j)$,
\begin{equation} \label{def_G}
\overline{U}_j^T (n \Delta t) = G(\Omega_j) \overline{U}_j^T \left( \left[ n-1 \right] \Delta t \right)
\end{equation}

Notar que $G = G(\Omega_j)$ es función de la discretización espacial. Reemplazando (\ref{solución transiente}) en (\ref{def_G}),

\begin{equation}
\overline{U}_j^0 e^{\Omega_j n \Delta t} = G(\Omega_j) \overline{U}_j^0 e^{\Omega_j (n-1) \Delta t}
\end{equation}

Despejando $G(\Omega_j)$ se obtiene el factor de amplicación para un paso de tiempo (de $ (n-1) \Delta t$ a $n \Delta t$)

\begin{equation}
G = e^{\Omega_j \Delta t}
\end{equation}

Entonces, el factor de amplificación $G$ calculado desde la solución inicial $U^0$ hasta el paso de tiempo $n \Delta t$ se obtiene,

\begin{equation}
G = e^{\Omega_j n \Delta t}
\end{equation}

Para garantizar que la solución transiente sea estable se debe cumplir que,

\begin{equation}
| G(\Omega_j) | = | e^{ \left[ \Re(\Omega_j) \right] + \left[ \Im(\Omega_j) n \Delta t \right] i} | < 1
\end{equation}

lo que implica que

\begin{equation}
\Re(\Omega_j) \leq 0
\end{equation}

Se desprende la ecuación anterior que $\Re(\Omega_j)$ esta asociado al error de disipación (exponencial), mientras que $\Im(\Omega_j)$ al error de dispersión (oscilación)



\newpage
%---------------------------------------------

\section{Desarrollo y Análisis} \label{DESARROLLO_Y_ANALISIS}

%----

\subsection{Ejercicios en Fortran}
\subsubsection{Ejercicio 1}

Sea $A(n)$ un número real tal que,
\begin{equation}
A(n) = \sum_{n=1}^\infty \dfrac{1}{n}
\end{equation}
Se implementa una programa en Fortran que permite calcular y graficar $A(n)$ para ciertos valores de $n$. Esta serie geométrica es divergente, es decir, $A(n) \rightarrow \infty$ cuando $n \rightarrow \infty$. \\

En la Figura \ref{fig_P1_1_1} se grafica la función $A$ en simple precisión y doble precisión ($A_{sp}$ y $A_{dp}$, respectivamente). El desarrollo de $A$ es práctimente el mismo, salvo una diferencia decimal, hasta aproximadamente $n = 100 000 $. Como se observa $A_{sp}$ y $A_{dp}$ empiezan a mostrar diferencias que se vuelven más prominentes en la medida que incrementa $n$. Para $n_{crit} = 2 097 151$, el valor de $A_{sp}$ se estanca, ya que,
$$ \dfrac{1}{n} = \dfrac{1}{2 097 151} \approx 0,000000477... $$
Los reales de simple precisión con los que trabaja Fortran poseen 7 cifras significativas, luego por redondeo $1/n_{crit}$ no contribuye a la sumatoria, resultando en $A_{sp} | _{n=k} = 15.4036827$ para $k \geq n_{crit} $. La diferencia entre los valores obtenidos con la simple y doble precision se muestran en la Figura \ref{fig_P1_1_2}, donde se grafica un error porcentual respecto a $A_{dp}$:
\begin{equation}
E(\%) = \dfrac{ A_{dp}(n) - A_{sp}(n) }{ A_{dp}(n) }
\end{equation}
donde se asume que el valor con doble precisión es el valor más exacto.\\

\begin{figure} [H]
\begin{center}
\includegraphics[width=0.8\textwidth]{./parte2/graficos/grafico_p1.pdf}
\caption{Gráfica de $A_{sp}$ y $A_{dp}$ para algunos valores de $n$. Abscisa en escala logarítmica} \label{fig_P1_1_1}
\end{center}
\end{figure}

\begin{figure} [H]
\begin{center}
\includegraphics[width=0.8\textwidth]{./parte2/graficos/grafico_p1_error.pdf}
\caption{Error relativo $E(\%)$ de $A_{sp}$ respecto a $A_{dp}$. Abscisa en escala logarítmica} \label{fig_P1_1_2}
\end{center}
\end{figure}

%-----------------------------------------------------------------------------
%-----------------------------------------------------------------------------
%-----------------------------------------------------------------------------

\subsubsection{Ejercicio 2}
Se implementa una rutina en Fortran que permite calcular los $n+1$ valores de la serie Fibonacci
\begin{equation}
u_{n+1} = u_{n} + u_{n-1} \hspace{1cm} \mbox{tal que} \hspace{0,5cm} u_0=0 \hspace{0,1cm} ; \hspace{0,1cm} u_1=1
\end{equation}

En las Figuras \ref{fig_P1_2_1} y \ref{fig_P1_2_2} se representa los elementos de la serie. Se observa en la primera gráfica que la serie presenta un crecimiento exponencial. Para $n=46$ se presenta una inestabilidad numérica, lo cual se explica por la memoria asignada a un objeto real (Tabla  \ref{TABLA_FORTRAN}). \\

Se utiliza un real de doble precision en vez de un entero y se grafica la serie (Figura \ref{fig_P1_2_real}). Se vuelve a apreciar el comportamiento exponencial. La serie se grafica hasta que el se alcanza el tope de memoria, arrojando \textit{inf}

\begin{figure} [H]
\begin{center}
\includegraphics[width=0.8\textwidth]{./parte2/graficos/grafico_p2_1.pdf}
\caption{Gráfica de la serie Fibonacci para $n \in [0, \ldots , 20]$. Ordenanda en escala logarítmica} \label{fig_P1_2_1}
\end{center}
\end{figure}

\begin{figure} [H]
\begin{center}
\includegraphics[width=0.8\textwidth]{./parte2/graficos/grafico_p2_2.pdf}
\caption{Gráfica de la serie Fibonacci para $n \in [0, \ldots , 100]$ definiendo $u$ como un objeto \texttt{integer}. Para $n=46$ se presenta una inestabilidad numérica: se obtiene números negativos en una serie estrictamente creciente. Ordenanda en escala logarítmica} \label{fig_P1_2_2}
\end{center}
\end{figure}

\begin{figure} [H]
\begin{center}
\includegraphics[width=0.8\textwidth]{./parte2/graficos/grafico_p2_real.pdf}
\caption{Gráfica de la serie Fibonacci para $n \in [0, \ldots , 1600]$ definiendo $u$ como un objeto \texttt{real} de doble precisión. El resultado crece exponencialmente hasta alcanzar el máximo de memoría permitido, arrojando \textit{inf}. Ordenanda en escala logarítmica} \label{fig_P1_2_real}
\end{center}
\end{figure}

%-----------------------------------------------------------------------------
%-----------------------------------------------------------------------------
%-----------------------------------------------------------------------------

\subsection{Ejercicio 3}

Se escribe una rutina \textit{matrix-mult} que realiza el producto entre dos matrices $\vect{A}$ y $\vect{B}$ de dimensiones $(m_a,n_a)$ y $(m_b,n_b)$ respectivamente. El algoritmo implementado es el siguiente:
\begin{enumerate}
\item Se ingresan los valores de las matrices $\vect{A}_{n_a,m_a}$ y $\vect{B}_{n_b,m_b}$.
\item Se verifica la dimensión entre $\vect{A}$ y $\vect{B}$. Si $m_a \neq n_b$ entonces $l=1$ y se sale de la subrutina. En el caso contrario, las dimensiones son consistente con la multiplicación matricial, $l=0$, y se pasa al siguiente paso ($l$: indicador del error)
\item Se define el tamaño y se asigna la memoria para arreglo $C_{n_a,m_b}$
\item Se realiza la multiplicación matricial entre $\vect{A}$ y $\vect{B}$. Se asigna a la matriz $\vect{C}$
\item Salida de la subrutina $\vect{C_}{n_a,m_b}$ y $l$
\end{enumerate}
En la Figura \ref{fig_P1_3} se expone un diagrama de flujo que explica la subrutina

\begin{figure} [H]
\begin{center}
\includegraphics[width=0.9\textwidth]{./parte2/graficos/diagrama_de_flujo.pdf}
\caption{Diagrama de flujo de la subrutina \textit{matrix-mult}. Variables de entrada: Matrices $\vect{A}_{n_a,m_a}$ y $\vect{B}_{n_b,m_b}$. Variables de salida: Matriz $\vect{C}_{n_a,m_b}$ y el indicador de error $l$} \label{fig_P1_3}
\end{center}
\end{figure}


%----

\subsection{Estudio del comportamiento mecánico de una arteria}
%Estudio del comportamiento mecánico de una arteria

\subsection{Parte 1: Movimiento de una pared arterial}

Una arteria puede modelarse por un cilindro flexible de base circular, longitud $L$, radio $R_0$, cuyas paredes poseen un espesor $H$. Se supone que está constituido de un material elástico, incompresible, homogéneo e isotrópico. \\

Un modelo simplificado que describe el comportamiento mecánico de la pared arterial en interacción con el flujo sanguíneo se obtiene considerando que el cilindro es constituido por un conjunto de anillos independientes uno de otros. De esta manera se puede despreciar las interacciones longitudinales y axiales a lo largo de la arteria. Luego, se supone que la arteria se deforma solamente en la dirección radial. \\

El radio de la arteria está dado por,
\begin{equation}
R(t) = R_0 + y(t)
\end{equation}
donde $y(t)$ es la deformación radial en función del tiempo $t$. Al aplicar la ley de Newton en el sistema de anillo independientes conduce a una ecuación que permite modelar el comportamiento mecánico de la pared de la arteria en función del tiempo,
\begin{equation} \label{PROBLEMA_PARTE2}
\dfrac{d^2 y(t)}{dt^2} + \beta \dfrac{dy(t)}{dt} + \alpha y(t) = \gamma (p(t)-p_0)
\end{equation}
donde,
\begin{equation}
\alpha = \dfrac{E}{\rho_w R^2_0} \hspace{0,5cm} \gamma = \dfrac{1}{\rho_w H} \hspace{0,5cm} \beta = \mbox{constante} > 0 
\end{equation}

Particularmente se modela la variación de la presión a lo largo de la arteria como una función sinusoidal que depende de la posición $x$ y el instante de tiempo $t$,
\begin{equation} \label{FUENTE_PARTE2}
(p-p_0) = x \Delta p  \left( a + b cos( \omega_0 t ) \right)
\end{equation} 

\subsubsection*{Simulación 1}

Se calcula numericamente la ecuación (\ref{PROBLEMA_PARTE2}) con el término fuente (\ref{FUENTE_PARTE2}).  Se uyilizan los siguientes valores realistas para los parámetros físicos:
\begin{table} [H]
\centering
\begin{tabular}{llll}
$L$		& = $5 \times 10 ^ {-2}$ m 				&	$b$	 		&= $133.32$ N m $^{-2}$ \\
$R_0$	& = $5 \times 10 ^ {-3}$ m 				&	$a$		 	&= $1333.2$ N m $^{-2}$ \\
$\rho_w$& = $1 \times 10 ^ {3}$ kg m $^{-3}$ 	&	$\Delta p$ 	&= $33.33$ N m $^{-2}$ \\
$H$		& = $3 \times 10 ^ {-4}$ m 				&	$w_0$ 		&= $2 \pi / 0.8$ \\
$E$		& = $9 \times 10 ^ {5}$ N m $^{-2}$ 	&				&
\end{tabular}
\caption{Parametros utilizados para la simulación 1} \label{PARAMETROS_PARTE2}
\end{table}

Y considerando a su vez dos parametros de $\beta$:
\begin{enumerate}[label=(\alph*)]
\item $\beta = \sqrt{ \alpha }$ 
\item $\beta = \alpha$
\end{enumerate}

Se reescribe la ecuación (\ref{PROBLEMA_PARTE2}) como un sustema de ecuaciones lineales. En forma matricial,

\begin{equation} \label{PROBLEMA_PARTE2_CORREGIDO}
\vec{y'}(t) = \vect{A} \vec{y} + \vec{b}
\end{equation}

donde $\vec{y} = \begin{vmatrix} y & y' \end{vmatrix}^T$ ($T$ significa transpuesta), y $\vec{b}(t)$ es un vector fuente dependiente del tiempo $t$. La matriz $\vect{A}$ resultante es,

\begin{equation}
\vect{A} = \begin{pmatrix}
0 & 1 \\ -\alpha & -\beta
\end{pmatrix}
\end{equation}

Los valores propios de $\vect{A}$ se obtienen del desarrollo del polinomio característico,

\begin{equation}
det(\vect{A} - \lambda \vect{I}) = \begin{vmatrix}
-\lambda & 1 \\
-\alpha & -\lambda -\beta 
\end{vmatrix} \rightarrow \alpha \lambda^2 + \beta \lambda + 1 = 0
\end{equation}

Luego, los valores propios se calculan como la raíz del polinomio,

\begin{equation}
\lambda_{1,2} = \dfrac{( -\beta \pm \sqrt{ \beta^2 - 4 \alpha})}{2}
\end{equation}

Notar que para valores de $\beta \geq 2\sqrt{\alpha}$ ambos valores, $\lambda_1$ y $\lambda_2$, resultan reales y negativos, mientras que para valores de $\beta < 2\sqrt{\alpha}$ ambos autovalores resultan números complejos con su componente real negativa. \\

Se implementa una subrutina que permite calcular los valores propios de la matriz $\vect{A}$. Utilizando los valores de la Tabla \ref{PARAMETROS_PARTE2} se obtiene:
\begin{enumerate} [label=(\alph*)]
\item $\beta = \sqrt{\alpha} = 6.0 \times 10^3 $
\begin{equation}
\vect{A} = \begin{pmatrix} 0 & 1 \\ 36.0 \times 10^6 & 6.0 \times 10^3 \end{pmatrix} \rightarrow 
\begin{matrix} \lambda_1 = & -3000.00 + 5196.15 i \\ \lambda_2 = & -3000.00 - 5196.15 i \end{matrix}
\end{equation} 
\item $\beta = \alpha =  36.0 \times 10^6 $
\begin{equation}
\vect{A} = \begin{pmatrix} 0 & 1 \\ 36.0 \times 10^6 & 36.0 \times 10^6 \end{pmatrix} \rightarrow 
\begin{matrix} \lambda_1 = & -1.0 \\ \lambda_2 = & -36.0 \times 10^6 \end{matrix}
\end{equation} 
\end{enumerate}

%------------------------------------

\subsubsection{Simulación 1: Euler Implicito} 

\paragraph{Discretización de la ecuación diferencial}
Se implementa una subrutina que permite calcular la ecuacion (\ref{PROBLEMA_PARTE2}) usando el método de Euler Implícito para dos valores de $\beta$. Sea $y(x,t) = y^n_j$ y $\partial y / \partial t (x,t) = z^n_j$, recurriendo a la expresión (\ref{PROBLEMA_PARTE2_CORREGIDO}) e implementando un esquema de integración implícito se tiene que,
\begin{align}
\dfrac{y^n - y^{n-1}}{\Delta t} &= z^n \\
\dfrac{z^n - z^{n-1}}{\Delta t} &= -\alpha y^n - \beta z^n + \gamma (p_n-p_0)
\end{align}

Reordenando los valores en los pasos de tiempo $n$ y $n-1$ en los lados izquierdo y derecho respectivamente, se expresa la relación anterior en forma matricial como,
\begin{equation}
\vect{A} \cdot \begin{Bmatrix}
y^n \\ z^n
\end{Bmatrix} =
\begin{Bmatrix}
y^{n-1} \\ z^{n-1}
\end{Bmatrix} +
\Delta t \begin{Bmatrix}
0 \\ \gamma (p_n-p_0)
\end{Bmatrix}
\end{equation}
donde,
\begin{equation}
\vect{A} = \begin{pmatrix}
1 & -\Delta t \\
\Delta t \alpha & 1+ \Delta t \beta
\end{pmatrix}
\end{equation}

Despenjando las incognitas $\begin{Bmatrix} y^n & z^n \end{Bmatrix} ^T$ se obtiene,
\begin{equation}
\begin{Bmatrix}
y^n \\ z^n
\end{Bmatrix} =
\vect{A}^{-1} \cdot \begin{Bmatrix}
y^{n-1} \\ z^{n-1}
\end{Bmatrix} +
\Delta t \vect{A}^{-1} \cdot  \begin{Bmatrix}
0 \\ \gamma (p_n-p_0)
\end{Bmatrix}
\end{equation}
donde
\begin{equation}
\vect{A}^{-1} = 
\dfrac{1}{1 + \beta \Delta t + \alpha (\Delta t) ^ 2}
\begin{pmatrix}
1+\beta \Delta t & \Delta t \\
-\Delta t \alpha & 1 
\end{pmatrix}
\end{equation}

\paragraph{Estabilidad de la solución}
Se quiere estudiar la estabilidad de la solución transiente de (\ref{PROBLEMA_PARTE2_CORREGIDO}). Para ello se recurre a las expresiones (\ref{analisis_espectral_general}) y (\ref{analisis_espectral_descompuesto}). Luego,
\begin{align}
\dfrac{ y^{n+1} - y^n } { \Delta t } & = \Omega_1 y^{n+1} \\
\dfrac{ z^{n+1} - z^n } { \Delta t } & = \Omega_2 z^{n+1}
\end{align} 

\begin{equation}
\dfrac{d \vec{U}}{d t} = \vect{S} \vec{U} + \vec{Q} \rightarrow \left\{ \begin{matrix}
( y^{n+1} - y^n ) /  \Delta t  = \Omega_1 y^{n+1} \\
( z^{n+1} - z^n ) /  \Delta t  = \Omega_2 z^{n+1}
\end{matrix} \right.
\end{equation}

Despejando los terminos evaluados en $t_{n+1}$ en la izquierda de la ecuación

\begin{align}
y^{n+1} &= \dfrac{ 1 }{ 1 - \Delta t \Omega_1 } y^n \\
z^{n+1} &= \dfrac{ 1 }{ 1 - \Delta t \Omega_2 } z^n
\end{align}

Se reconocen los términos $\vec{z}_p$ para $y$ y $z$. Teniendo en cuenta los valores propios $\Omega$ antes calculado

\begin{align}
z_p &= \dfrac{1}{1-\Omega_j \Delta t} \\
&= \dfrac{1}{ \left[ 1-\Re(\Omega_j) \Delta t \right] - \left[ \Im(\Omega_j) \Delta t \right] i }\\
&= \dfrac{\left[ 1- \Re(\Omega_j) \Delta t \right] + \left[ \Im(\Omega_j) \Delta t \right] i }{ \left[ 1-\Re(\Omega_j) \Delta t \right]^2 + \left[ \Im(\Omega_j) \Delta t \right]^2}
\end{align}

El módulo de $z_p$ viene dado por

\begin{equation}
||z_p|| = \dfrac{ \sqrt{ \left[ 1-\Re(\Omega_j) \Delta t \right]^2 + \left[ \Im(\Omega_j) \Delta t \right]^2 } } { \left[ 1- \Re(\Omega_j) \Delta t \right]^2 + \left[ \Im(\Omega_j) \Delta t \right]^2}
\end{equation}

\begin{enumerate}[label=(\alph*)]

\item $\beta = \alpha$ y $\Delta t = 10^4$
\begin{center}
\begin{tabular}{lll}
$\Omega_1 = −3000.00 + 5196.15i$ & $\rightarrow$ & $z_{p1} = $ \\
$\Omega_2 = −3000.00 - 5196.15i$ & $\rightarrow$ & $z_{p2} = $
\end{tabular}
\end{center}

\item $\beta = \sqrt{\alpha}$ y $\Delta t = 10^4$
\begin{center}
\begin{tabular}{lll}
$\Omega_1 = -1.0$ & $\rightarrow$ & $z_{p1} = $ \\
$\Omega_2 = -36.0 \times 10^6$ & $\rightarrow$ & $z_{p2} = $
\end{tabular}
\end{center}

\end{enumerate}



%----------------------- FIGURAS SIMULACION 1 -----------------------------
%----------------------- 	EULER IMPLICITO  -----------------------------

\begin{center}
\begin{figure} [H]
	\begin{subfigure}[b]{0.3\textwidth}
		\includegraphics{./parte3/graficos/grafico_euler_S1_y_b1.pdf}
		\caption{} 
		\label{fig:eulerS1b1_y}
	\end{subfigure}
	
	\begin{subfigure}[b]{0.3\textwidth}
		\includegraphics{./parte3/graficos/grafico_euler_S1_dy_b1.pdf}
		\caption{} 
		\label{fig:eulerS1b1_dy}
	\end{subfigure}
\caption{} \label{euler_S1_b1}
\end{figure}
\end{center}


\begin{center}
\begin{figure} [H]
	\begin{subfigure}[b]{0.3\textwidth}
		\includegraphics{./parte3/graficos/grafico_euler_S1_y_b2.pdf}
		\caption{} 
		\label{fig:eulerS1b2_y}
	\end{subfigure}
	
	\begin{subfigure}[b]{0.3\textwidth}
		\includegraphics{./parte3/graficos/grafico_euler_S1_dy_b2.pdf}
		\caption{} 
		\label{fig:eulerS1b2_dy}
	\end{subfigure}
\caption{} \label{euler_S1_b2}
\end{figure}
\end{center}

%----------------------------------------------------------------------------

\subsubsection{Simulación 1: Crank Nicolson}

Discretizacion de la ecuación por el método Crank Nicolson

\begin{align}
\dfrac{y^{n+1}-y^n}{\Delta t} & = \dfrac{1}{2} \left( z^{n+1} + z^n \right) \\
\dfrac{z^{n+1}-z^n}{\Delta t} &= \dfrac{1}{2} \left( -\alpha y^{n+1} - \beta y^{n+1} + \gamma (p_{n+1}-p_0) \right) + \dfrac{1}{2} \left( -\alpha y^{n} - \beta y^{n} + \gamma (p_{n}-p_0) \right)  
\end{align}

La relación anterior se escribe en forma matricial,

\begin{equation}
\vect{A} \cdot
\begin{Bmatrix}
y^{n+1} \\ z^{n+1}
\end{Bmatrix} =
\vect{B} \cdot
\begin{Bmatrix}
y^n \\ z^n
\end{Bmatrix} + \dfrac{\Delta t}{2}
\begin{Bmatrix}
0 \\ \left( \gamma (p_{n+1}-p_0) + \gamma (p_{n}-p_0) \right)
\end{Bmatrix}
\end{equation}

donde,

\begin{equation}
\vect{A} = \begin{pmatrix}
1 & -\dfrac{\Delta t}{2} \\
\dfrac{\alpha \Delta t}{2} & 1+ \dfrac{\beta \Delta t}{2}
\end{pmatrix} 
\end{equation}

\begin{equation}
\vect{B} = \begin{pmatrix}
1 & \dfrac{\Delta t}{2} \\
-\dfrac{\alpha \Delta t}{2} & 1-\dfrac{\beta \Delta t}{2}
\end{pmatrix}
\end{equation}

Despejando las variables incognitas se obtiene,

\begin{equation}
\begin{Bmatrix}
y^{n+1} \\ z^{n+1} 
\end{Bmatrix} =
\vect{A}^{n-1} \cdot \vect{B} \cdot 
\begin{Bmatrix}
y^n \\ z^n 
\end{Bmatrix} + \dfrac{\Delta t}{2} \vect{A}^{-1} \cdot
\begin{Bmatrix}
0 \\ \left( \gamma (p_{n+1}-p_0) + \gamma (p_{n}-p_0) \right)
\end{Bmatrix}
\end{equation}

donde 

\begin{equation}
\vect{A}^{-1} = \dfrac{1}{1 + \beta \dfrac{\Delta t}{2} + \alpha \left( \dfrac{\Delta t}{2} \right)^2 } 
\begin{pmatrix}
1 + \dfrac{\beta \Delta t}{2} & \dfrac{\Delta t}{2}\\
-\dfrac{\alpha \Delta t}{2} & 1 
\end{pmatrix}
\end{equation}

Se grafican los resultados y se exponen en las Figura \ref{cn_S1_b1} y \ref{cn_S1_b2}

Se quiere estudiar la estabilidad de la solución transiente (homogénea) de (QUE ECUACIÓN). Para ello se ...

\begin{align}
\dfrac{ y^{n+1} - y^n } { \Delta t } & = \Omega_1 \dfrac{1}{2} \left( y^{n+1} + y^{n} \right) \\
\dfrac{ z^{n+1} - z^n } { \Delta t } & = \Omega_2 \dfrac{1}{2} \left( z^{n+1} + z^{n} \right)
\end{align} 

Despejando los terminos evaluados en $t_{n+1}$ en la izquierda de la ecuación

\begin{align}
y^{n+1} &= \dfrac{ 1 + \Delta t \Omega_1 }{ 1 - \Delta t \Omega_1 } y^n \\
z^{n+1} &= \dfrac{ 1 + \Delta t \Omega_2 }{ 1 - \Delta t \Omega_2 } z^n
\end{align}

Se reconocen los términos $\vec{z}_p$ para $y$ y $z$. Teniendo en cuenta los valores propios $\Omega$ antes calculado

\begin{align}
z_p &= \dfrac{1+\Omega_j \Delta t}{1-\Omega_j \Delta t / 2} \\
&= \dfrac{\left[ 1-\Re(\Omega_j) \Delta t / 2 \right] + \left[ \Im(\Omega_j) \Delta t / 2 \right] i}
		 {\left[ 1- \Re(\Omega_j) \Delta t / 2 \right] - \left[ \Im(\Omega_j) \Delta t / 2 \right] i} \\
&= \dfrac{\left[ \left( 1- \Re(\Omega_j) \Delta t / 2 \right) + \left( \Im(\Omega_j) \Delta t / 2 \right) i \right] \cdot \left[ \left( 1+ \Re(\Omega_j) \Delta t / 2 \right) + \left( \Im(\Omega_j) \Delta t / 2 \right) i \right]}
		 {\left[ 1- \Re(\Omega_j) \Delta t / 2 \right]^2 + \left[ \Im(\Omega_j) \Delta t / 2 \right]^2} \\
&= \dfrac{ \left[ 1 - \Re(\Omega_j)\Delta t / 2 - \Im(\Omega_j)\Delta t / 2 \right] + \left[ \Im (\Omega_j) \Delta t / 2 \right] i  } {\left[ 1-\Delta t \Re(\Omega_j) \right]^2 + \left[ \Delta t \Im(\Omega_j) \right]^2}
\end{align}

El módulo de $z_p$ viene dado por

\begin{equation}
||z_p|| = \dfrac{ \sqrt{ \left[ 1 - \dfrac{\Re(\Omega_j)\Delta t}{2} - \dfrac{\Im(\Omega_j)\Delta t}{2} \right]^2 + \left[ \dfrac{\Im \Delta t}{2}\right]^2 } } {\left[ \dfrac{1-\Delta t \Re(\Omega_j)}{2} \right]^2 + \left[ \dfrac{\Delta t \Im(\Omega_j)}{2} \right]^2}
\end{equation}

\begin{enumerate}[label=(\alph*)]

\item $\beta = \alpha$ y $\Delta t = 10^4$
\begin{center}
\begin{tabular}{lll}
$\Omega_1 = −3000.00 + 5196.15i$ & $\rightarrow$ & $z_{p1} = $ \\
$\Omega_2 = −3000.00 - 5196.15i$ & $\rightarrow$ & $z_{p2} = $
\end{tabular}
\end{center}

\item $\beta = \sqrt{\alpha}$ y $\Delta t = 10^4$
\begin{center}
\begin{tabular}{lll}
$\Omega_1 = -1.0$ & $\rightarrow$ & $z_{p1} = $ \\
$\Omega_2 = -36.0 \times 10^6$ & $\rightarrow$ & $z_{p2} = $
\end{tabular}
\end{center}

\end{enumerate}

%----------------------- FIGURAS SIMULACION 1 -----------------------------
%----------------------- 	CRANK NICOLSON -----------------------------

\begin{center}
\begin{figure} [H]
	\begin{subfigure}[b]{0.3\textwidth}
		\includegraphics{./parte3/graficos/grafico_cn_S1_y_b1.pdf}
		\caption{} 
		\label{fig:cnS1b1_y}
	\end{subfigure}
	
	\begin{subfigure}[b]{0.3\textwidth}
		\includegraphics{./parte3/graficos/grafico_cn_S1_dy_b1.pdf}
		\caption{} 
		\label{fig:cnS1b1_dy}
	\end{subfigure}
\caption{} \label{cn_S1_b1}
\end{figure}
\end{center}

\begin{center}
\begin{figure} [H]
	\begin{subfigure}[b]{0.3\textwidth}
		\includegraphics{./parte3/graficos/grafico_cn_S1_y_b2.pdf}
		\caption{} 
		\label{fig:cnS1b2_y}
	\end{subfigure}
	
	\begin{subfigure}[b]{0.3\textwidth}
		\includegraphics{./parte3/graficos/grafico_cn_S1_dy_b2.pdf}
		\caption{} 
		\label{fig:cnS1b2_dy}
	\end{subfigure}
\caption{} \label{cn_S1_b2}
\end{figure}
\end{center}

%-------------------------------------------------------------------------

\subsubsection{Simulación 2}

%----------------------- FIGURAS SIMULACION 2 -----------------------------
%----------------------- 	EULER IMPLICITO  -----------------------------

\begin{center}
\begin{figure} [H]
	\begin{subfigure}[b]{0.3\textwidth}
		\includegraphics{./parte3/graficos/grafico_euler_S2_y_b1.pdf}
		\caption{} 
		\label{fig:eulerS2b1_y}
	\end{subfigure}
	
	\begin{subfigure}[b]{0.3\textwidth}
		\includegraphics{./parte3/graficos/grafico_euler_S2_dy_b1.pdf}
		\caption{} 
		\label{fig:eulerS2b1_dy}
	\end{subfigure}
\end{figure}
\end{center}

\begin{center}
\begin{figure} [H]
	\begin{subfigure}[b]{0.3\textwidth}
		\includegraphics{./parte3/graficos/grafico_euler_S2_y_b2.pdf}
		\caption{} 
		\label{fig:eulerS2b2_y}
	\end{subfigure}
	
	\begin{subfigure}[b]{0.3\textwidth}
		\includegraphics{./parte3/graficos/grafico_euler_S2_dy_b2.pdf}
		\caption{} 
		\label{fig:eulerS2b2_dy}
	\end{subfigure}
\end{figure}
\end{center}

%--------------------------------------------------------------------------


%----------------------- FIGURAS SIMULACION 2 -----------------------------
%-----------------------	CRANK NICOLSON -----------------------------

\begin{center}
\begin{figure} [H]
	\begin{subfigure}[b]{0.3\textwidth}
		\includegraphics{./parte3/graficos/grafico_cn_S2_y_b1.pdf}
		\caption{} 
		\label{fig:cnS2b1_y}
	\end{subfigure}
	
	\begin{subfigure}[b]{0.3\textwidth}
		\includegraphics{./parte3/graficos/grafico_cn_S2_dy_b1.pdf}
		\caption{} 
		\label{fig:cnS2b1_dy}
	\end{subfigure}
\end{figure}
\end{center}

\begin{center}
\begin{figure} [H]
	\begin{subfigure}[b]{0.3\textwidth}
		\includegraphics{./parte3/graficos/grafico_cn_S2_y_b2.pdf}
		\caption{} 
		\label{fig:cnS2b2_y}
	\end{subfigure}
	
	\begin{subfigure}[b]{0.3\textwidth}
		\includegraphics{./parte3/graficos/grafico_cn_S2_dy_b2.pdf}
		\caption{} 
		\label{fig:cnS2b2_dy}
	\end{subfigure}
\end{figure}

\end{center}

%------------------------------------------------------------------------

COMENTAR RESULTADOS Y ESTUDIAR SI ES POSIBLE UTILIZAR UNA SOLUCIÓN 
%Estudio del comportamiento mecánico de una arteria

\subsection{Parte 2: Un modelo hiperbólico para la interacción de la sangre con la pared}

Si para el mismo fenomeno descrito en la Parte 1 no se desprecia la interacción axial entre los anillos, entonces la ecuación (\ref{PROBLEMA_PARTE2}) se modifica resultado en,
\begin{equation} \label{PROBLEMA_PARTE_2_2}
\rho_\omega H \dfrac{\partial^2 y}{\partial t^2} - \sigma_x \dfrac{\partial^2 y}{\partial x^2} + \dfrac{H \, E}{R_0^2} y = p-p_0 \hspace{1cm} t>0 \hspace{1cm} 0<x<L
\end{equation}
Se denota la coordenada longitudinal $x$. $\sigma_x$ es la componente radial del esfuerzo axial y $L$ es el largo del cilindro considerado. Despreciando el factor de $y$ de la ecuación (\ref{PROBLEMA_PARTE_2_2}) y considerando $p-p_0 = f$ entonces se obtiene la ecuación de onda en una dimensión
\begin{equation} \label{E_ONDA}
\dfrac{\partial^2 u}{\partial t^2} - \gamma \dfrac{\partial^2 u}{\partial x^2} = f \hspace{1cm} x \in \left] \alpha , \beta \right[ \hspace{1cm} t>0
\end{equation}

Se emplean los esquemas de Leap-Frog y Newmark para discretizar la ecuación anterior.

\subsubsection{Leap-Frog}

El termino fuente utilizado para la simulación es $ f = ( 1 + \pi^2 \gamma^2 ) e^{-t} sin( \pi x ) $. Empleando un esquema de diferencias centradas en el espacio,

\begin{equation}
\dfrac{ \partial^2 u }{ \partial t^2 } = \gamma^2 \dfrac{ u_{j+1}^n - 2 u_j^n + u_{j-1}^n }{ \Delta x ^2} + 
 ( 1 + \pi^2 \gamma^2 ) e^{-t} sin( \pi x_j ) 
\end{equation}
Se utiliza el esquema de integración temporal Leap-Frog,
\begin{equation}
u^{n+1}_j - 2u^n_j + u^{n-1} = (\gamma \lambda)^2 ( u^n_{j-1} -2 u^n_j + u^n_{j+1} ) + f^n_j
\end{equation}
donde $\lambda= \Delta t / \Delta x$

\paragraph{Discretización espacial}

Se realiza una descomposición modal como se expuso en la Sección \ref{analisis_espectral_seccion},
\begin{align*}
S e ^ { i k_j p \Delta x } &= \dfrac{\gamma^2}{\Delta x^2} \left( e ^ { i k_j (p+1) \Delta x } - 2 e ^ { i k_j p \Delta x } + e ^ { i k_j (p-1) \Delta x } \right) \\
&= \underbrace{\dfrac{2 \gamma^2}{\Delta x^2} \left( cos(k_j \Delta x) - 1 \right)}_{\Omega_j} e ^ { i k_j p \Delta x}
\end{align*}
Entonces, los valores propios $\Omega_j$ son,
\begin{equation} \label{OMEGA_LF}
\Omega_j = \dfrac{2 \gamma^2}{\Delta x^2} \left( cos(k_j \Delta x) - 1 \right)
\end{equation}
Notar que $\Omega_j = \Re(\Omega_j)$. Se debe cumplir que:
\begin{equation*}
\Re(\Omega_j) \leq 0
\end{equation*}
\begin{equation}
\dfrac{2 \gamma^2 }{ \Delta x^2 } \left[ cos(k_j \Delta x) -1 ) \right] \leq 0
\end{equation}
Finalmente se debe cumplir que,
\begin{equation}
\dfrac{-4 \gamma^2}{\Delta x^2} \leq \Re(\Omega_j) \leq 0
\end{equation}

\paragraph{Discretización temporal}

Estudia la estabilidad en el tiempo: se reemplaza $u_{j} = \omega$
\begin{equation}
\dfrac{\omega^{n+1} -2\omega^{n} + \omega^{n+1}}{\Delta t^2} = \vect{S} = \Omega_j \omega^n
\end{equation}
Reeordenando,
\begin{equation}
\omega^{n+1} - (\Omega_j \Delta t^2 +2) \omega^n + \omega^{n-1}=0
\end{equation}
Sea $z_p \approx G(\Omega_j)$ una aproximación del factor de amplificación, se puede escribir la ecuación anterior como,
\begin{equation}
z_p^2 \omega^{n} - z_p (\Omega_j \Delta t^2 +2)  \omega^n + \omega^{n-1} = 0
\end{equation}
Se deduce entonces,
\begin{equation}
z_p^2 - \left( \Omega_j \Delta t^2 +2 \right) z_p + 1 = 0
\end{equation}
Diviendo por $z_p$ ($z_p \neq 0$  ya que se está trabajando en estado transiente)
\begin{equation}
\left( \Omega_j \Delta t^2 + 2 \right) = \dfrac{1}{z_p} + z_p
\end{equation}
Se impone como solución $z_p = e^{i \theta}$ (notar que $|z_p| \leq 1$) . Reemplazando en la ecuación anterior,
\begin{equation}
\left( \Omega_j \Delta t^2 + 2 \right) = e^{i \theta} + e^{-i \theta} = 2 cos(\theta)
\end{equation}
La estabilidad se consigue acotando el lado izquierdo de la ecuación, obteniendo
\begin{equation}
-4 \leq \Re(\Omega_j \Delta t^2) \leq 0 
\end{equation}
Reemplazando $\Omega_j$ obtenido de la ecuación (\ref{OMEGA_LF}) y considerando $\Omega_j = \Re (\Omega_j)$ resulta,

\begin{equation}
-4 \leq \dfrac{\gamma^2 \Delta t^2}{\Delta x^2} \left( cos(k_j \Delta x) - 1 \right)  \leq 0
\end{equation}

El criterio de estabilidad es

\begin{equation}
\dfrac{\gamma^2 \Delta t^2}{\Delta x^2} \leq 2
\end{equation}

$\gamma^2 \Delta t^2 / \Delta x^2 $ se puede interpretar como $(CFL)^2$ donde $CFL$ es el número de Courant-Friedrichs-Lewy

\subsubsection{Newmark}

De la ecuacion (\ref{newmark_v}) (Sección \ref{esquema_newmark_seccion}) se tiene $\partial u / \partial t = v$
\begin{equation}
\dfrac{\partial v}{\partial t} = \gamma ^2 \left[ \Theta \left( \dfrac{ u_{j+1}^{n+1} -2u_{j}^{n+1} + u_{j-1}^{n+1} }{\Delta x^2} \right) - (1-\Theta) \left( \dfrac{ u_{j+1}^{n} -2u_{j}^{n} + u_{j-1}^{n} }{\Delta x^2} \right) \right]
\end{equation} 

\paragraph{Discretización espacial}

Se utiliza $\Theta=0.5$ (Esquema Crank Nicolson)
\begin{equation}
\dfrac{\partial v}{\partial t} = \left[ \left( \vect{S} e^{i k_j p \Delta x} \right)^{n+1} + \left( \vect{S} e^{i k_j p \Delta x} \right)^{n}  \right]
\end{equation}
Se resulve $(\vect{S})^n$
\begin{align*}
S e ^ { i k_j p \Delta x } &= \dfrac{\gamma^2}{2 \Delta x^2} \left( e ^ { i k_j (p+1) \Delta x } - 2 e ^ { i k_j p \Delta x } + e ^ { i k_j (p-1) \Delta x } \right) \\
&= \underbrace{\dfrac{\gamma^2}{\Delta x^2} \left( cos(k_j \Delta x) - 1 \right)}_{\Omega_j} e ^ { i k_j p \Delta x}
\end{align*}
Es decir,
\begin{equation} \label{OMEGA_j}
\Omega_j = \dfrac{\gamma^2}{\Delta x^2} \left( cos(k_j \Delta x) - 1 \right)
\end{equation}
\begin{equation} \label{CONDICION_OMEGA_j_NM}
\dfrac{-2 \gamma^2}{\Delta x^2} \leq \Re(\Omega_j) \leq 0
\end{equation}

Se obtiene el mismo resultado para $(\vect{S})^{n+1}$, Sumando $(\vect{S})^{n+1} + (\vect{S})^{n}$ luego obtenemos la mismas ecuación obtenida en la discretización Leap-Frog. Por lo tanto, la condición de $\Re(\Omega_j')$ ($\Omega_j' = \Omega_j^{n+1} + \Omega_j^{n}$)
\begin{equation}
\dfrac{-4 \gamma^2}{\Delta x^2} \leq \Re(\Omega_j') \leq 0
\end{equation}
Notar que $u_j$ y $v_j$ emplean diferencias finitas centradas para discretizar el dominio, por lo tanto se obtienen los mismos valores propios de $\vect{S}$

\paragraph{Discretización temporal}

Se integra la ecuación $\partial v / \partial t$. Se utiliza la notación $v_j = \omega$

\begin{equation}
\dfrac{\omega^{n+1}-\omega^n}{\Delta t} = \Omega_j^{n+1} \omega^{n+1} + \Omega_j^{n} \omega^{n}
\end{equation}

agrupando términos

\begin{equation}
\omega^{n+1} = \underbrace{ \dfrac{1+\Omega_j \Delta t}{1-\Omega_j \Delta t} }_{z_p} \omega^n
\end{equation}

como $\Omega_j = \Re(\Omega_j)$, entonces

\begin{equation} \label{zp_nm_v}
z_p = \dfrac{ 1 + \Omega_j }{ 1-\Omega_j}
\end{equation}

tomando en cuenta la condición (\ref{CONDICION_OMEGA_j_NM}) se verifica que

\begin{equation}
-1 \leq z_p \leq 1
\end{equation}

\subsubsection{Resultados}
A continuación se muestra la convergencia de los métodos Leap-Frog y Newmark
\begin{table} [H]
\begin{center}
\begin{tabular}{|llll|llll|} \hline
$t_j^{(0)}$ & $p_{LF}^{(1)}$ & $p_{LF}^{(2)}$ & $p_{LF}^{(3)}$ & $t_j^{(0)}$ & $p_{NW}^{(1)}$ & $p_{NW}^{(2)}$ & $p_{NW}^{(3)}$  \\ \hline
   0.1000  & -1.8233  & -1.7515  & -1.2447   &  0.1000  &  2.5229  &  2.0236  &  2.3471  \\
   0.2000  & -1.3451  & -1.1519  & -0.5895   &  0.2000  &  1.6501  &  1.7153  &  2.2387  \\
   0.3000  & -0.7807  & -0.5283  &  0.0635   &  0.3000  &  1.3800  &  1.6463  &  2.2604  \\
   0.4000  &  0.1544  &  0.4775  &  1.1064   &  0.4000  &  1.2057  &  1.6738  &  2.3914  \\
   0.5000  &  2.6457  &  3.7409  &  5.3562   &  0.5000  &  0.8952  &  1.9875  &  3.2940  \\
   0.6000  &  1.8799  &  2.1175  &  2.7069   &  0.6000  &  1.6377  &  1.5219  &  2.0791  \\
   0.7000  &  4.5569  &  4.8068  &  4.6839   &  0.7000  &  1.3065  &  2.4637  &  4.5019  \\
   0.8000  &  1.5724  &  1.7390  &  2.2901   &  0.8000  &  1.3260  &  0.9585  &  1.3992  \\
   0.9000  &  1.2640  &  1.5152  &  2.1067   &  0.9000  &  1.3014  &  1.4944  &  2.0995  \\
   1.0000  &  1.7824  &  2.1298  &  2.7728   &  1.0000  &  1.3536  &  1.8813  &  2.6445  \\
 \hline
\end{tabular}
\caption{} \label{tabla_lf_nm}
\end{center}
\end{table}




%----

\subsection{Atractor de Lorenz}
%ATRACTOR DE LORENZ
El sistema de ecuaciones de Lorenz es un ejemplo de un sistema de ecuaciones diferenciales de orden 1, tridimensional, no lineal, que tiene un comportamiento caótico para algunos valores de sus parámetros. Este sistema de ecuaciones permite modelar los rollos de convección que se producen en la atmosfera terrestre. Es un modelo simplificado de la convección de Rayleigh-Benard (ecuaciones de Navier-Stokes con hipótesis de Boussinesq).
\begin{equation}
\left\{ 
\begin{matrix}
dx/dt =& Pr (y(t) - x(yt)) \\
dy/dt =& Ra x(t) - y(t) - x(t)z(t) \\
dz/dt =& x(t) y(t) - \beta z(t)
\end{matrix} \right.
\end{equation}
Donde $Pr$ es el número de Prandtl y $Ra$ es el número de Rayleigh. Las variables dinámicas $x$, $y$ y $z$ represetan el estado del sistema a cada instante $t$:

\begin{itemize}
\item x(t) es proporcional a la intensidad del movimiento de convección  
\item y(t) es proporcional a la diferencia de temperatura entre las corrientes ascendentes y descendentes
\item z(t) es proporcional a la diferencia entre el perfil vertical de temperatura y un perfil vertical de temperatura lineal
\end{itemize}

Los sistemas dinámicos son sistemas que son función del tiempo. Que sea caótico significa que varia de manera no lineal y a su vez que el sistema presenta sensibilidad a frente a los parámetros entrada. Además presentan un comportamiento oscilante, pudiendo ser periodico o no periodico. \\

\subsubsection{Parte 1}
Las Figuras \ref{lorenz1} y \ref{lorenz2} reproducen los gráficos del articulo \textit{Deterministic Nonperiodic Flow} de Lorenz
\begin{figure} [H]
\begin{center}
\includegraphics[width=0.8\textwidth]{./parte4/graficos/FIGURA1.pdf}
\caption{} \label{lorenz1}
\end{center}
\end{figure}

\begin{figure} [H]
\begin{center}
\includegraphics[width=0.8\textwidth]{./parte4/graficos/FIGURA2.pdf}
\caption{} \label{lorenz2}
\end{center}
\end{figure}

%---------------------------

\subsubsection{Parte 2}

Se grafica la solución a la ecuación de convección utilizando los siguientes valores: $Pr=10$, $\beta= 8/3$ y $Ra = 0.5 \, , \, 10 , \, 28$ 

\begin{figure} [H]
\begin{center}
\includegraphics[width=0.8\textwidth]{./parte4/graficos/grafico_P3_3d_ra05.pdf}
\caption{}
\end{center}
\end{figure}

\begin{figure} [H]
\begin{center}
\includegraphics[width=0.8\textwidth]{./parte4/graficos/grafico_P3_3d_ra10.pdf}
\caption{}
\end{center}
\end{figure}

\begin{figure} [H]
\begin{center}
\includegraphics[width=0.8\textwidth]{./parte4/graficos/grafico_P3_3d_ra28.pdf}
\caption{}
\end{center}
\end{figure}

%---------------------------

\subsubsection{Parte 3}
Haciendo variar paulatinamente $Ra$ entre 0 y 30

\begin{figure} [H]
\hspace{-1cm} \includegraphics[width=0.6\textwidth]{./parte4/graficos/grafico_P3_3d_ra00.pdf}\includegraphics[width=0.6\textwidth]{./parte4/graficos/grafico_P3_3d_ra03.pdf}
\caption{} 
\end{figure}

\begin{figure} [H]
\hspace{-1cm} \includegraphics[width=0.6\textwidth]{./parte4/graficos/grafico_P3_3d_ra06.pdf}\includegraphics[width=0.6\textwidth]{./parte4/graficos/grafico_P3_3d_ra09.pdf}
\caption{} 
\end{figure}

\begin{figure} [H]
\hspace{-1cm} \includegraphics[width=0.6\textwidth]{./parte4/graficos/grafico_P3_3d_ra12.pdf}\includegraphics[width=0.6\textwidth]{./parte4/graficos/grafico_P3_3d_ra15.pdf}
\caption{} 
\end{figure}

\begin{figure} [H]
\hspace{-1cm} \includegraphics[width=0.6\textwidth]{./parte4/graficos/grafico_P3_3d_ra18.pdf}\includegraphics[width=0.6\textwidth]{./parte4/graficos/grafico_P3_3d_ra18.pdf}
\caption{} 
\end{figure}

\begin{figure} [H]
\hspace{-1cm} \includegraphics[width=0.6\textwidth]{./parte4/graficos/grafico_P3_3d_ra21.pdf}\includegraphics[width=0.6\textwidth]{./parte4/graficos/grafico_P3_3d_ra24.pdf}
\caption{} 
\end{figure}

\begin{figure} [H]
\hspace{-1cm} \includegraphics[width=0.6\textwidth]{./parte4/graficos/grafico_P3_3d_ra27.pdf}\includegraphics[width=0.6\textwidth]{./parte4/graficos/grafico_P3_3d_ra30.pdf}
\caption{} 
\end{figure}

\newpage
%---------------------------------------------

\section{Conclusiones y Observaciones}

\newpage
%---------------------------------------------

\begin{thebibliography}{3}

\bibitem{lorenz} \textsc{Lorenz, E.} , \textit{Deterministic Nonperiodic Flow}, Journal of the Atmospheric Science, vol. 20, pag. 130-141, 1963

\bibitem{fortran} \textsc{Chapman, S.} , \textit{Fortran 90/95 for Scientits and Engineers}, Ediforial Mcgraw-Hill, 2003, ISBN-13: 978-0072922387

\bibitem{num_mat} \textsc{Quarteroni, A.} , \textsc{Sacco, R.} y \textsc{Saleri, F.} , \textit{Numerical Mathematics},
Editorial Springer, 2000, ISBN 0-387-98959-5

\bibitem{met_num} \textsc{Chapra, S.} y \textsc{Canale, R.} , \textit{Métodos numéricos para ingenieros}, 5ta edición,
Editorial McGraw-Hill, 2007, ISBN-13: 978-970-10-6114-5
\end{thebibliography}

\end{document}
