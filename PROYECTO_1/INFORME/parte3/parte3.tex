%Estudio del comportamiento mecánico de una arteria

\subsection{Parte 1: Movimiento de una pared arterial}

Una arteria puede modelarse por un cilindro flexible de base circular, longitud $L$, radio $R_0$, cuyas paredes poseen un espesor $H$. Se supone que está constituido de un material elástico, incompresible, homogéneo e isotrópico. 

Un modelo simplificado que describe el comportamiento mecánico de la pared arterial en interacción con el flujo sanguíneo se obtiene considerando que el cilindro es constituido por un conjunto de anillos independientes uno de otros. De esta manera se puede despreciar las interacciones longitudinales y axiales a lo largo de la arteria. Luego, se supone que la arteria se deforma solamente en la dirección radial.

El radio de la arteria está dado por,

\begin{equation}
R(t) = R_0 + y(t)
\end{equation}

donde $y(t)$ es la deformación radial en función del tiempo $t$. Al aplicar la ley de Newton en el sistema de anillo independientes conduce a una ecuación que permite modelar el comportamiento mécanico de la pared de la arteria en función del tiempo,

\begin{equation} \label{PROBLEMA_PARTE2}
\dfrac{d^2 y(t)}{dt^2} + \beta \dfrac{dy(t)}{dt} + \alpha y(t) = \gamma (p(t)-p_0)
\end{equation}

donde,

\begin{equation}
\alpha = \dfrac{E}{\rho_w R^2_0} \hspace{0,5cm} \gamma = \dfrac{1}{\rho_w H} \hspace{0,5cm} \beta = \mbox{constante} > 0 
\end{equation}

Particularmente se modela la variación de la presión a lo largo de la arteria como una función sinusoidal que depende de la posición $x$ y el instante de tiempo $t$,

\begin{equation} \label{FUENTE_PARTE2}
(p-p_0) = x \Delta p  \left( a + b cos( \omega_0 t ) \right)
\end{equation} 

\subsubsection{Simulación 1}

Se calcula numericamente la ecuación \ref{PROBLEMA_PARTE2} con el término fuente \ref{FUENTE_PARTE2} y considerando los siguientes valores realistas para los parámetros físicos:
\begin{table}
\centering
\begin{tabular}{llll}
$L$		& = $5 \times 10 ^ {-2}$ m 				&	$b$	 		&= $133.32$ N m $^{-2}$ \\
$R_0$	& = $5 \times 10 ^ {-3}$ m 				&	$a$		 	&= $1333.2$ N m $^{-2}$ \\
$\rho_w$& = $1 \times 10 ^ {3}$ kg m $^{-3}$ 	&	$\Delta p$ 	&= $33.33$ N m $^{-2}$ \\
$H$		& = $3 \times 10 ^ {-4}$ m 				&	$w_0$ 		&= $2 \pi / 0.8$ \\
$E$		& = $9 \times 10 ^ {5}$ N m $^{-2}$ 	&				&
\end{tabular}
\caption{Parametros utilizados para la simulación 1} \label{PARAMETROS_PARTE2}
\end{table}

Y considerando a su vez dos parametros de $\beta$:
\begin{enumerate}[label=(\alph*)]
\item $\beta = \sqrt{ \alpha }$ 
\item $\beta = \alpha$
\end{enumerate}

Se reescribe la ecuación (\ref{PROBLEMA_PARTE2}) como un sustema de ecuaciones lineales. En forma matricial,

\begin{equation} \label{PROBLEMA_PARTE2_CORREGIDO}
\vec{y'}(t) = \vect{A} \vec{y} + \vec{b}
\end{equation}

donde $\vec{y} = \begin{vmatrix} y & y' \end{vmatrix}^T$ ($T$ significa transpuesta), y $\vec{b}(t)$ es un vector fuente dependiente del tiempo $t$. La matriz $\vect{A}$ resultante es,

\begin{equation}
\vect{A} = \begin{pmatrix}
0 & 1 \\ -\alpha & -\beta
\end{pmatrix}
\end{equation}

Se implementa una subrutina que permite calcular los valores propios de la matriz $\vect{A}$ de la ecuación \ref{PROBLEMA_PARTE2_CORREGIDO}. Utilizando los valores de la Tabla \ref{PARAMETROS_PARTE2} se obtiene:
\begin{enumerate} [label=(\alph*)]
\item $\beta = \sqrt{\alpha} = 6.0 \times 10^3 $
\begin{equation}
\vect{A} = \begin{pmatrix} 0 & 1 \\ 36.0 \times 10^6 & 6.0 \times 10^3 \end{pmatrix} \rightarrow 
\begin{matrix} \lambda_1 = & -3000.00 + 5196.15 i \\ \lambda_2 &= -3000.00 - 5196.15 i \end{matrix}
\end{equation} 
\item $\beta = \alpha =  36.0 \times 10^6 $
\begin{equation}
\vect{A} = \begin{pmatrix} 0 & 1 \\ 36.0 \times 10^6 & 6.0 \times 10^3 \end{pmatrix} \rightarrow 
\begin{matrix} \lambda_1 = & -3000.00 + 5196.15 i \\ \lambda_2 &= -3000.00 - 5196.15 i \end{matrix}
\end{equation} 
\end{enumerate}

Se implementa una subrutina que permite calcular la ecuacion (AGREGAR ECUACION) usando el método de Euler Implícito para dos valores de $\beta$ 



%------------------------------------

HABLAR DE LOS VALORES PROPIOS

%------------------------------------

Discretizacion de la ecuación por el método Euler Implicito
\begin{align}
\dfrac{y^n - y^{n-1}}{\Delta t} &= z^n \\
\dfrac{z^n - z^{n-1}}{\Delta t} &= -\alpha y^n - \beta z^n + \gamma (p_n-p_0) \\
\end{align}

Reordenando los valores en los pasos de tiempo $n$ $n-1$ en el lado izquierdo y derecho respectivamente, se expresa la relacion (ECUACION ANTERIOR) en forma matricial como,
\begin{equation}
\vect{A} \cdot \begin{Bmatrix}
y^n \\ z^n
\end{Bmatrix} =
\begin{Bmatrix}
y^{n-1} \\ z^{n-1}
\end{Bmatrix} +
\Delta t \begin{Bmatrix}
0 \\ \gamma (p_n-p_0)
\end{Bmatrix}
\end{equation}
donde
\begin{equation}
\vect{A} = \begin{pmatrix}
1 & -\Delta t \\
\Delta t \alpha & 1+ \Delta t \beta
\end{pmatrix}
\end{equation}


Despenjando las incognitas se obtiene,
\begin{equation}
\begin{Bmatrix}
y^n \\ z^n
\end{Bmatrix} =
\vect{A}^{-1} \cdot \begin{Bmatrix}
y^{n-1} \\ z^{n-1}
\end{Bmatrix} +
\Delta t \vect{A}^{-1} \cdot  \begin{Bmatrix}
0 \\ \gamma (p_n-p_0)
\end{Bmatrix}
\end{equation}
donde
\begin{equation}
\vect{A}^{-1} = 
\dfrac{1}{1 + \beta \Delta t + \alpha (\Delta t) ^ 2}
\begin{pmatrix}
1+\beta \Delta t & \Delta t \\
-\Delta t \alpha & 1 
\end{pmatrix}
\end{equation}

%------------------------

Discretizacion de la ecuación por el método Crank Nicolson

\begin{align}
\dfrac{y^{n+1}-y^n}{\Delta t} & = \dfrac{1}{2} \left( z^{n+1} + z^n \right) \\
\dfrac{z^{n+1}-z^n}{\Delta t} &= \dfrac{1}{2} \left( -\alpha y^{n+1} - \beta y^{n+1} + \gamma (p_{n+1}-p_0) \right) + \dfrac{1}{2} \left( -\alpha y^{n} - \beta y^{n} + \gamma (p_{n}-p_0) \right)  
\end{align}

Reordenando los valores en los pasos de tiempo $n$ $n-1$ en el lado izquierdo y derecho respectivamente, se expresa la relacion (ECUACION ANTERIOR) en forma matricial como,

\begin{equation}
\vect{A} \cdot
\begin{Bmatrix}
y^{n+1} \\ z^{n+1}
\end{Bmatrix} =
\vect{B} \cdot
\begin{Bmatrix}
y^n \\ z^n
\end{Bmatrix} + \dfrac{\Delta t}{2}
\begin{Bmatrix}
0 \\ \left( \gamma (p_{n+1}-p_0) + \gamma (p_{n}-p_0) \right)
\end{Bmatrix}
\end{equation}

donde
\begin{equation}
\vect{A} = \begin{pmatrix}
1 & -\dfrac{\Delta t}{2} \\
\dfrac{\alpha \Delta t}{2} & 1+ \dfrac{\beta \Delta t}{2}
\end{pmatrix} 
\end{equation}

\begin{equation}
\vect{B} = \begin{pmatrix}
1 & \dfrac{\Delta t}{2} \\
-\dfrac{\alpha \Delta t}{2} & 1-\dfrac{\beta \Delta t}{2}
\end{pmatrix}
\end{equation}

Despejando las variables incognitas se obtiene,

\begin{equation}
\begin{Bmatrix}
y^{n+1} \\ z^{n+1} 
\end{Bmatrix} =
\vect{A}^{n-1} \cdot \vect{B} \cdot 
\begin{Bmatrix}
y^n \\ z^n 
\end{Bmatrix} + \dfrac{\Delta t}{2} \vect{A}^{-1} \cdot
\begin{Bmatrix}
0 \\ \left( \gamma (p_{n+1}-p_0) + \gamma (p_{n}-p_0) \right)
\end{Bmatrix}
\end{equation}

donde 
\begin{equation}
\vect{A}^{-1} = \dfrac{1}{1 + \beta \dfrac{\Delta t}{2} + \alpha \left( \dfrac{\Delta t}{2} \right)^2 } 
\begin{pmatrix}
1 + \dfrac{\beta \Delta t}{2} & \dfrac{\Delta t}{2}\\
-\dfrac{\alpha \Delta t}{2} & 1 
\end{pmatrix}
\end{equation}

%-----------------------
\subsection{Parte 2: Un modelo hiperbólico para la interacción de la sangre con la pared}

\begin{equation}
\dfrac{\partial^2 u}{\partial t^2} - \gamma^2 \dfrac{\partial^2 u}{\partial x^2} = f \hspace{0,5cm} x \in ] \alpha , \beta [
\end{equation}

Esquema de discretización de Leapfrog
\begin{equation}
u^{n+1}_j - 2u^n_j + u^{n-1} = (\gamma \lambda)^2 ( u^n_{j-1} -2 u^n_j + u^n_{j+1} ) + f^n_j
\end{equation}
donde $\lambda=\dfrac{\Delta t}{\Delta x}$
