%Estudio del comportamiento mecánico de una arteria

\subsection{Parte 2: Un modelo hiperbólico para la interacción de la sangre con la pared}

Si para el mismo fenomeno descrito en la Parte 1 no se desprecia la interacción axial entre los anillos, entonces la ecuación (\ref{PROBLEMA_PARTE2}) se modifica resultado en,
\begin{equation} \label{PROBLEMA_PARTE_2_2}
\rho_\omega H \dfrac{\partial^2 y}{\partial t^2} - \sigma_x \dfrac{\partial^2 y}{\partial x^2} + \dfrac{H \, E}{R_0^2} y = p-p_0 \hspace{1cm} t>0 \hspace{1cm} 0<x<L
\end{equation}
Se denota la coordenada longitudinal $x$. $\sigma_x$ es la componente radial del esfuerzo axial y $L$ es el largo del cilindro considerado. Despreciando el factor de $y$ de la ecuación (\ref{PROBLEMA_PARTE_2_2}) y considerando $p-p_0 = f$ entonces se obtiene la ecuación de onda en una dimensión
\begin{equation} \label{E_ONDA}
\dfrac{\partial^2 u}{\partial t^2} - \gamma \dfrac{\partial^2 u}{\partial x^2} = f \hspace{1cm} x \in \left] \alpha , \beta \right[ \hspace{1cm} t>0
\end{equation}

Se emplean los esquemas de Leap-Frog y Newmark para discretizar la ecuación anterior.

\subsubsection{Leap-Frog}

El termino fuente utilizado para la simulación es $ f = ( 1 + \pi^2 \gamma^2 ) e^{-t} sin( \pi x ) $. Empleando un esquema de diferencias centradas en el espacio,

\begin{equation}
\dfrac{ \partial^2 u }{ \partial t^2 } = \gamma^2 \dfrac{ u_{j+1}^n - 2 u_j^n + u_{j-1}^n }{ \Delta x ^2} + 
 ( 1 + \pi^2 \gamma^2 ) e^{-t} sin( \pi x_j ) 
\end{equation}
Se utiliza el esquema de integración temporal Leap-Frog,
\begin{equation}
u^{n+1}_j - 2u^n_j + u^{n-1} = (\gamma \lambda)^2 ( u^n_{j-1} -2 u^n_j + u^n_{j+1} ) + f^n_j
\end{equation}
donde $\lambda= \Delta t / \Delta x$

\paragraph{Discretización espacial}

Se realiza una descomposición modal como se expuso en la Sección \ref{analisis_espectral_seccion},
\begin{align*}
S e ^ { i k_j p \Delta x } &= \dfrac{\gamma^2}{\Delta x^2} \left( e ^ { i k_j (p+1) \Delta x } - 2 e ^ { i k_j p \Delta x } + e ^ { i k_j (p-1) \Delta x } \right) \\
&= \underbrace{\dfrac{2 \gamma^2}{\Delta x^2} \left( cos(k_j \Delta x) - 1 \right)}_{\Omega_j} e ^ { i k_j p \Delta x}
\end{align*}
Entonces, los valores propios $\Omega_j$ son,
\begin{equation} \label{OMEGA_LF}
\Omega_j = \dfrac{2 \gamma^2}{\Delta x^2} \left( cos(k_j \Delta x) - 1 \right)
\end{equation}
Notar que $\Omega_j = \Re(\Omega_j)$. Se debe cumplir que:
\begin{equation*}
\Re(\Omega_j) \leq 0
\end{equation*}
\begin{equation}
\dfrac{2 \gamma^2 }{ \Delta x^2 } \left[ cos(k_j \Delta x) -1 ) \right] \leq 0
\end{equation}
Finalmente se debe cumplir que,
\begin{equation}
\dfrac{-4 \gamma^2}{\Delta x^2} \leq \Re(\Omega_j) \leq 0
\end{equation}

\paragraph{Discretización temporal}

Estudia la estabilidad en el tiempo: se reemplaza $u_{j} = \omega$
\begin{equation}
\dfrac{\omega^{n+1} -2\omega^{n} + \omega^{n+1}}{\Delta t^2} = \vect{S} = \Omega_j \omega^n
\end{equation}
Reeordenando,
\begin{equation}
\omega^{n+1} - (\Omega_j \Delta t^2 +2) \omega^n + \omega^{n-1}=0
\end{equation}
Sea $z_p \approx G(\Omega_j)$ una aproximación del factor de amplificación, se puede escribir la ecuación anterior como,
\begin{equation}
z_p^2 \omega^{n} - z_p (\Omega_j \Delta t^2 +2)  \omega^n + \omega^{n-1} = 0
\end{equation}
Se deduce entonces,
\begin{equation}
z_p^2 - \left( \Omega_j \Delta t^2 +2 \right) z_p + 1 = 0
\end{equation}
Diviendo por $z_p$ ($z_p \neq 0$  ya que se está trabajando en estado transiente)
\begin{equation}
\left( \Omega_j \Delta t^2 + 2 \right) = \dfrac{1}{z_p} + z_p
\end{equation}
Se impone como solución $z_p = e^{i \theta}$ (notar que $|z_p| \leq 1$) . Reemplazando en la ecuación anterior,
\begin{equation}
\left( \Omega_j \Delta t^2 + 2 \right) = e^{i \theta} + e^{-i \theta} = 2 cos(\theta)
\end{equation}
La estabilidad se consigue acotando el lado izquierdo de la ecuación, obteniendo
\begin{equation}
-4 \leq \Re(\Omega_j \Delta t^2) \leq 0 
\end{equation}
Reemplazando $\Omega_j$ obtenido de la ecuación (\ref{OMEGA_LF}) y considerando $\Omega_j = \Re (\Omega_j)$ resulta,

\begin{equation}
-4 \leq \dfrac{\gamma^2 \Delta t^2}{\Delta x^2} \left( cos(k_j \Delta x) - 1 \right)  \leq 0
\end{equation}

El criterio de estabilidad es

\begin{equation}
\dfrac{\gamma^2 \Delta t^2}{\Delta x^2} \leq 2
\end{equation}

$\gamma^2 \Delta t^2 / \Delta x^2 $ se puede interpretar como $(CFL)^2$ donde $CFL$ es el número de Courant-Friedrichs-Lewy

\subsubsection{Newmark}

De la ecuacion (\ref{newmark_v}) (Sección \ref{esquema_newmark_seccion}) se tiene $\partial u / \partial t = v$
\begin{equation}
\dfrac{\partial v}{\partial t} = \gamma ^2 \left[ \Theta \left( \dfrac{ u_{j+1}^{n+1} -2u_{j}^{n+1} + u_{j-1}^{n+1} }{\Delta x^2} \right) - (1-\Theta) \left( \dfrac{ u_{j+1}^{n} -2u_{j}^{n} + u_{j-1}^{n} }{\Delta x^2} \right) \right]
\end{equation} 

\paragraph{Discretización espacial}

Se utiliza $\Theta=0.5$ (Esquema Crank Nicolson)
\begin{equation}
\dfrac{\partial v}{\partial t} = \left[ \left( \vect{S} e^{i k_j p \Delta x} \right)^{n+1} + \left( \vect{S} e^{i k_j p \Delta x} \right)^{n}  \right]
\end{equation}
Se resulve $(\vect{S})^n$
\begin{align*}
S e ^ { i k_j p \Delta x } &= \dfrac{\gamma^2}{2 \Delta x^2} \left( e ^ { i k_j (p+1) \Delta x } - 2 e ^ { i k_j p \Delta x } + e ^ { i k_j (p-1) \Delta x } \right) \\
&= \underbrace{\dfrac{\gamma^2}{\Delta x^2} \left( cos(k_j \Delta x) - 1 \right)}_{\Omega_j} e ^ { i k_j p \Delta x}
\end{align*}
Es decir,
\begin{equation} \label{OMEGA_j}
\Omega_j = \dfrac{\gamma^2}{\Delta x^2} \left( cos(k_j \Delta x) - 1 \right)
\end{equation}
\begin{equation} \label{CONDICION_OMEGA_j_NM}
\dfrac{-2 \gamma^2}{\Delta x^2} \leq \Re(\Omega_j) \leq 0
\end{equation}

Se obtiene el mismo resultado para $(\vect{S})^{n+1}$, Sumando $(\vect{S})^{n+1} + (\vect{S})^{n}$ luego obtenemos la mismas ecuación obtenida en la discretización Leap-Frog. Por lo tanto, la condición de $\Re(\Omega_j')$ ($\Omega_j' = \Omega_j^{n+1} + \Omega_j^{n}$)
\begin{equation}
\dfrac{-4 \gamma^2}{\Delta x^2} \leq \Re(\Omega_j') \leq 0
\end{equation}
Notar que $u_j$ y $v_j$ emplean diferencias finitas centradas para discretizar el dominio, por lo tanto se obtienen los mismos valores propios de $\vect{S}$

\paragraph{Discretización temporal}

Se integra la ecuación $\partial v / \partial t$. Se utiliza la notación $v_j = \omega$

\begin{equation}
\dfrac{\omega^{n+1}-\omega^n}{\Delta t} = \Omega_j^{n+1} \omega^{n+1} + \Omega_j^{n} \omega^{n}
\end{equation}

agrupando términos

\begin{equation}
\omega^{n+1} = \underbrace{ \dfrac{1+\Omega_j \Delta t}{1-\Omega_j \Delta t} }_{z_p} \omega^n
\end{equation}

como $\Omega_j = \Re(\Omega_j)$, entonces

\begin{equation} \label{zp_nm_v}
z_p = \dfrac{ 1 + \Omega_j }{ 1-\Omega_j}
\end{equation}

tomando en cuenta la condición (\ref{CONDICION_OMEGA_j_NM}) se verifica que

\begin{equation}
-1 \leq z_p \leq 1
\end{equation}

\subsubsection{Resultados}
A continuación se muestra la convergencia de los métodos Leap-Frog y Newmark
\begin{table} [H]
\begin{center}
\begin{tabular}{|llll|llll|} \hline
$t_j^{(0)}$ & $p_{LF}^{(1)}$ & $p_{LF}^{(2)}$ & $p_{LF}^{(3)}$ & $t_j^{(0)}$ & $p_{NW}^{(1)}$ & $p_{NW}^{(2)}$ & $p_{NW}^{(3)}$  \\ \hline
   0.1000  & -1.8233  & -1.7515  & -1.2447   &  0.1000  &  2.5229  &  2.0236  &  2.3471  \\
   0.2000  & -1.3451  & -1.1519  & -0.5895   &  0.2000  &  1.6501  &  1.7153  &  2.2387  \\
   0.3000  & -0.7807  & -0.5283  &  0.0635   &  0.3000  &  1.3800  &  1.6463  &  2.2604  \\
   0.4000  &  0.1544  &  0.4775  &  1.1064   &  0.4000  &  1.2057  &  1.6738  &  2.3914  \\
   0.5000  &  2.6457  &  3.7409  &  5.3562   &  0.5000  &  0.8952  &  1.9875  &  3.2940  \\
   0.6000  &  1.8799  &  2.1175  &  2.7069   &  0.6000  &  1.6377  &  1.5219  &  2.0791  \\
   0.7000  &  4.5569  &  4.8068  &  4.6839   &  0.7000  &  1.3065  &  2.4637  &  4.5019  \\
   0.8000  &  1.5724  &  1.7390  &  2.2901   &  0.8000  &  1.3260  &  0.9585  &  1.3992  \\
   0.9000  &  1.2640  &  1.5152  &  2.1067   &  0.9000  &  1.3014  &  1.4944  &  2.0995  \\
   1.0000  &  1.7824  &  2.1298  &  2.7728   &  1.0000  &  1.3536  &  1.8813  &  2.6445  \\
 \hline
\end{tabular}
\caption{} \label{tabla_lf_nm}
\end{center}
\end{table}


